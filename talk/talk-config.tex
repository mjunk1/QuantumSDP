% ********************************************************************    
% Packages
% ********************************************************************
\usepackage[utf8]{inputenc}
\usepackage{default}
\usepackage{tikz}
\usepackage{tikz-3dplot}
\usepackage{pgfplots}
% \usepackage{tikz-3dplot}
\usepackage{graphicx}
\usepackage{algpseudocode}
\usepackage{subfig}
\usepackage{amsmath,amssymb}
\usepackage{mathtools}
\usepackage{braket}
\usepackage{hyperref}
\usepackage{slashed}
\usepackage{pifont}
\usepackage{appendixnumberbeamer}
\usepackage{tcolorbox}

\usepackage[backend=bibtex8,citestyle=authoryear,isbn=false,url=false,doi=false]{biblatex}
\addbibresource{Bibliography.bib}

% ********************************************************************    
% Settings
% ********************************************************************

% biblatex
\DeclareCiteCommand{\myfootcite}
  {\footnotetext{\usebibmacro{prenote}}
  {\usebibmacro{citeindex}%
    \usebibmacro{author}
    \printfield{journaltitle}
    \printfield{volume}
    (\printfield{year})}
  {\multicitedelim}
  {\usebibmacro{postnote}}}

% beamer
\usetheme{Copenhagen}
\usecolortheme{beaver}
% \usefonttheme{structuresmallcapsserif}
\setbeamertemplate{navigation symbols}{}
\setbeamertemplate{footline}[frame number]
\setbeamertemplate{section in toc}[square]
% \setbeamercolor{section in toc}{use=structure,fg=structure.fg}
% \setbeamercolor{section in toc}{fg=blue}

% additional settings
% \hypersetup{colorlinks=true}
% \everymath{\displaystyle}

\tcbset{%
%     noparskip,
    colback=gray!10, %background color of the box
    colframe=gray!40, %color of frame and title background
    coltext=black, %color of body text
    coltitle=black, %color of title text 
    fonttitle=\bfseries,
%     alerted/.style={coltitle=red, 
%                      colframe=gray!40},
%     example/.style={coltitle=black, 
%                      colframe=green!20,             
%                      colback=green!5},
    }

% tikz & pgf
\usetikzlibrary{shapes.misc}
\usetikzlibrary{shapes,snakes}
\usetikzlibrary{arrows}
% \tdplotsetmaincoords{70}{30}

% for external nodes
\tikzstyle{every picture}+=[remember picture]

% \usepgfplotslibrary{external}
% \tikzexternalize[prefix=gfx/]
% \tikzset{external/force remake}

\pgfplotsset{compat=1.14}

% ********************************************************************    
% Commands
% ********************************************************************

\newcommand{\cmark}{\ding{51}}%
\newcommand{\xmark}{\ding{55}}%
\newcommand{\done}{\rlap{$\square$}{\raisebox{2pt}{\large\hspace{1pt}\cmark}}%
\hspace{-2.5pt}}

% meta
\author{Daniel Brosch \and Markus Heinrich}
\title{Quantum Semidefinite Programming}
\date{
  Convex Optimization Seminar \\
  Bad Honnef \\
  August 7 -- 9, 2017
}
% \institute{
%   Convex Optimization Seminar, Bad Honnef
% }

% abbreviations
\newcommand{\ie}{i.\,e.}
\newcommand{\Ie}{I.\,e.}
\newcommand{\eg}{e.\,g.}
\newcommand{\Eg}{E.\,g.} 

% ********************************************************************    
% Math definitions
% ********************************************************************
% sets
\newcommand{\R}{\mathbb{R}}
\newcommand{\N}{\mathbb{N}}
\newcommand{\C}{\mathbb{C}}
\newcommand{\Z}{\mathbb{Z}}

% MathOperators
\DeclareMathOperator{\Ric}{Ric}
\DeclareMathOperator{\Tr}{Tr}
\DeclareMathOperator{\sgn}{sgn}
\DeclareMathOperator{\Con}{Con}
\DeclareMathOperator{\oracle}{ORACLE}

% operators
\newcommand{\id}{\mathrm{id}}
\newcommand{\dd}{\,\mathrm{d}}

% groups
\newcommand{\Orth}{O}
\newcommand{\SOrth}{SO}
\newcommand{\orth}{\mathfrak{O}}
\newcommand{\sorth}{\mathfrak{so}}
\newcommand{\Conf}{\mathrm{CO}}
\newcommand{\conf}{\mathfrak{co}}
\newcommand{\U}{U}
\newcommand{\SU}{SU}

% styling and text
\newcommand{\vect}[1]{{\boldsymbol{#1}}}
\newcommand{\const}{\text{const}}
\newcommand{\Op}{\mathcal{O}}
\newcommand{\T}{\mathbb{T}}
\newcommand{\Sp}{\mathbb{Sp}}
\newcommand{\A}{\mathcal{A}}
\newcommand{\V}{\mathcal{V}}
\newcommand{\F}{\mathcal{F}}
\newcommand{\G}{\mathcal{G}}
\newcommand{\D}{\mathcal{D}}
\newcommand{\f}{\tilde{f}}
\newcommand{\aM}{\overline{a}}
\newcommand{\TM}{\overline{T}}
\newcommand{\JM}{\overline{J}_y}

%\newcommand{\slashed}[1]{\ensuremath{\mathrlap{\!\not{\phantom{#1}}}#1}}% \fsl{<symbol>}
\newcommand{\slashedd}[1]{#1\!\!\!/}